\subsection{OGC API - Features}
\begin{frame}
  \frametitle{\currentsectionname}

  \begin{itemize}
    \item API-Standard für Geodaten vom Open Geospatial Consortium.
    \item Wird von Simplex4TwIS genutzt, um Daten des Realitätsmodells und SimplexSzenarios auszugeben.
    \item Für den Block-Editor werden Teile des Entwurfs "OGC API - Features - Part 3: Filtering" genutzt:
          \begin{itemize}
            \item Common Query Language (CQL)
            \item Funktionen
            \item Queryables
          \end{itemize}
    % \item Objekt- und Verbindungsklassen im Realitätsmodell werden in der API als \textit{collections} ausgegeben, Objekte und Verbindungen sind deren \textit{items}.
  \end{itemize}

  \Note{
    \item auch Importtabellen können über die API als collection ausgelesen werden
  }

\end{frame}

% \begin{frame}
%   \frametitle{Common Query Language (CQL)}
%   \begin{itemize}
%     \item Eingeführt zur Angabe von Filterbedingungen bei Anfragen von OGC API-Dienste.
%     \item Inhaltlich analog zur \texttt{WHERE}-Klausel in SQL.
%     \item Unterstützt boolsche Prädikate, Textvergleiche, zeitliche und raumbezogen Datentypen und Operatoren.
%   \end{itemize}
%
%   \Note{
%     \item Textversion und JSON
%     \item Kann um Prädikate, Operatoren und Datentypen erweitert werden
%   }
%
% \end{frame}
%
% \begin{frame}
%   \frametitle{Funktionen}
%
%   \begin{itemize}
%     \item Serverseitige Möglichkeit, CQL zu erweitern.
%     \item Syntax ähnelt typischen Programmier- und Abfragesprachen.
%     \item API informiert unter \texttt{/functions}-Endpunkt über verfügbare Funktionen, inklusive Name, Parameter- und Rückgabetypen.
%   \end{itemize}
%
%   \Note{
%     \item Syntax ähnelt jedenfalls in der TEXT-Version von CQL
%     \item optional können auch weitere Metadaten wie Beschreibungen zur Funktion und zu den einzelnen Parametern ausgegeben werden
%     \item Block-Editor nutzt diese Resource, um die verfügbaren Funktionen abzufragen und aufzulisten
%   }
%
% \end{frame}
%
% \begin{frame}
%   \frametitle{Queryables}
%
%   \begin{itemize}
%     \item Alle abfragbaren Felder einer Resource im API-Features-Dienst.
%     \item Für jede collection können so Metadaten abgefragt werden.
%     \item In Simplex4TwIS haben somit Importtabellen und Objektklassen eine gemeinsame Schnittstelle, über die die interne Struktur bekannt gegeben wird.
%   \end{itemize}
%
%   \Note{
%     \item gemeinsame Schnittstelle wichtig für Block-Editor, da sowohl Importtabellen als auch Objektklassen als Input für Editor genutzt werden können
%   }
%
% \end{frame}
