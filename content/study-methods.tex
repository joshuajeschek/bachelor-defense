\subsection{Usability-Studie}
\begin{frame}
  \frametitle{\currentsectionname}

  \begin{itemize}
    \item Formative Studie mit 10 Teilnehmer:innen.
    \item 6 Angestellte und 4 Kund:innen.
    \item 3 Szenarios, bei denen immer neue Funktionalitäten genutzt werden müssen.
    \item Qualititative Daten:
          \begin{itemize}
            \item Concurrent-Think-Aloud-Protokoll
            \item Nachgespräch mit vorbereiteten Fragen als Gedankenanstoß
          \end{itemize}
    \item Quantitative Daten:
          \begin{itemize}
            \item Single Ease Question nach jedem Szenario
            \item Einschätzung und Vergleich zur Vorversion auf einer Skala von 1 bis 10.
          \end{itemize}
  \end{itemize}

  \Note{
    \item sowohl mit Programmierkenntnissen als auch ohne
    \item entspricht Zielgruppe, da Anwendung sowohl von Angestellten als auch Kund:innen genutzt wird
  }

\end{frame}
