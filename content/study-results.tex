\subsection{Erkenntnisse}

\begin{frame}
  \frametitle{\currentsectionname}

  \begin{itemize}
    \item Der Editor wurde als Vereinfachung der Abläufe empfunden.
    \item Tipp- und Syntax-Fehler konnten vermieden werden, semantische Fehler traten selten auf.
    \item Alle Teilnehmer:innen fühlten sich mit dem Block-Editor effizienter als zuvor, da z.B. häufiges Nachschlagen entfällt.
  \end{itemize}

  \Note{
    \item CTA hat meistens gut funktioniert
    \item eine Person äußerte sich freudig darüber nun solche Aufgaben auch selbst durchführen zu können
    \item Grund für Effizienz auch Auswahlmenü, verhindert häufiges Nachschlagen
  }

\end{frame}

\begin{frame}
  % \frametitle{\currentsectionname{} (cont.)}

  \begin{figure}
    \pgfplotstableread[col sep=comma]{assets/study-results.csv}\datatable
\centering
\resizebox{.95\textheight}{!}{%
  \begin{tikzpicture}
    \begin{axis}[
        boxplot/draw direction=y,
        boxplot/every median/.style={ultra thick},
        xtick={1,2,3},
        xticklabels={Szenario 1, Szenario 2, Szenario 3},
        ytick={1,...,7},
        ymin=1,
        ylabel={SEQ},
      ]
      \addplot+ [boxplot, plot1] table [y=a] {\datatable};
      \addplot+ [boxplot, plot2] table [y=b] {\datatable};
      \addplot+ [boxplot, plot3] table [y=c] {\datatable};
    \end{axis}
  \end{tikzpicture}
}

    \caption{Antworten auf die Single Ease Question nach jedem Szenario (1: sehr schwer, 7: sehr einfach).}
  \end{figure}

  \Note{
    \item bevor wir auf Details eingehen erstmal nachschauen was so die quantitativen Daten sagen, die haben dann meistens zu Gesprächen geführt die ins Detail gehen
    \item erstes Szenario einfache Zuordnung, wurde als am einfachsten eingeschätzt
    \item Szenario 2 und 3 ungefähr ähnlich eingeschätzt, aber bei Szenario 3 gingen Meinungen weiter auseinander, könnte an unterschiedlichen Erfahrungsständen und Vertrautheit mit Konzept von SimplexSzenario liegen
  }

\end{frame}

\begin{frame}
  % \frametitle{\currentsectionname{} (cont.)}

  \begin{figure}
    \pgfplotstableread[col sep=comma]{assets/study-results.csv}\datatable
\centering
\resizebox{.95\textheight}{!}{%
  \begin{tikzpicture}
    \begin{axis}[
        boxplot/draw direction=y,
        boxplot/every median/.style={ultra thick},
        xtick={1,2},
        xticklabels={Vorversion, Block-Editor},
        ytick={1,...,10},
        ymin=1,
        ylabel={Einfachheit der Benutzung},
      ]
      \addplot+ [boxplot, plot1] table [y=prev] {\datatable};
      \addplot+ [boxplot, plot2] table [y=ges] {\datatable};
    \end{axis}
  \end{tikzpicture}
}

    \caption{Einschätzung der Einfachheit der Benutzung nach dem Usability Test auf einer Skala von 1 (sehr schwer) bis 10 (sehr einfach).}
  \end{figure}

  \Note{
    \item deutlich zu erkennen, dass Block-Editor besser als die Vorversion eingeschätzt wird
  }

\end{frame}

\begin{frame}
  % \frametitle{\currentsectionname{} (cont.)}

  \begin{figure}
    \colorlet{presentation}{plot1}
\colorlet{interaction}{plot2}
\colorlet{content}{plot3}
\colorlet{technical}{plot4}
\centering
\resizebox{.75\textheight}{!}{%
  \begin{tikzpicture}
    \begin{axis}[
        xbar=0pt,
        xmajorgrids=true,
        xtick={0,...,10},
        xmin=0,
        xmax=6,
        xlabel={Absolute Häufigkeit},
        /pgf/bar shift=0pt,
        legend style={legend cell align=left},
        legend pos=south east,
        axis y line*=none,
        axis x line*=bottom,
        tick label style={font=\footnotesize},
        legend style={font=\footnotesize},
        label style={font=\footnotesize},
        width=.6\textwidth,
        bar width=3.5mm,
        ymin=1,
        ytick={1,...,20},
        ytick style={draw=none},
        yticklabels={
            {Übersichtlichkeit Funktionsliste},
            {Präfix von Objektklassen},
            {Details zu Funktionen},
            {Benennung Abfragbare Felder},
            {Probleme mit Scrollen},
            {Zugriff auf die Quelldaten},
            {Automatische Typ-Filterung},
            {Angabe von statischen Werten},
            {Auswahl von Überladungen},
            {Benennung rechter Bereich},
            {Benennung des Speichern-Buttons},
            {Parameterübernahme bei Überladungen},
            {Drag and Drop},
            {Ersetzen von Einträgen},
            {Details zu Parametern},
            {Metadaten von Szenario-Feldern},
            {Anzeige von Attributen in Ziel},
            {Datentyp von Szenario-Feldern},
            {Symbole für Datentypen},
            {Bedienreihenfolge von Funktionen},
          },
        area legend,
        y=6mm,
        enlarge y limits={abs=0.625},
        every axis plot/.append style={fill}
      ]
      \addplot[interaction]  coordinates {(0,0)};  \addlegendentry{Interaktion (8)}
      \addplot[presentation] coordinates {(0,0)};  \addlegendentry{Darstellung (7)}
      \addplot[content]      coordinates {(0,0)};  \addlegendentry{Inhalt (4)}
      \addplot[technical]    coordinates {(0,0)};  \addlegendentry{Technisch (1)}

      \addplot[presentation] coordinates {(1,1)};  % Übersichtlichkeit Funktionsliste
      \addplot[content]      coordinates {(2,2)};  % Präfix von Objektklassen
      \addplot[content]      coordinates {(2,3)};  % Details zu Funktionen
      \addplot[presentation] coordinates {(2,4)};  % Benennung Abfragbare Felder
      \addplot[technical]    coordinates {(3,5)};  % Probleme mit Scrollen
      \addplot[content]      coordinates {(3,6)};  % Zugriff auf die Quelldaten
      \addplot[interaction]  coordinates {(3,7)};  % automatische Typ-Filterung
      \addplot[interaction]  coordinates {(3,8)};  % Angabe von statischen Werten
      \addplot[interaction]  coordinates {(3,9)};  % Auswahl von Überladungen
      \addplot[presentation] coordinates {(4,10)}; % Benennung rechter Bereich
      \addplot[presentation] coordinates {(4,11)}; % Benennung des Speichern-Buttons
      \addplot[interaction]  coordinates {(4,12)}; % Parameterübernahme bei Überladungen
      \addplot[interaction]  coordinates {(4,13)}; % Drag and Drop
      \addplot[interaction]  coordinates {(4,14)}; % Ersetzen von Einträgen
      \addplot[content]      coordinates {(5,15)}; % Details zu Parametern
      \addplot[presentation] coordinates {(5,16)}; % Metadaten von Szenario-Feldern
      \addplot[presentation] coordinates {(5,17)}; % Anzeige von Attributen in Ziel
      \addplot[interaction]  coordinates {(5,18)}; % Datentyp von Szenario-Feldern
      \addplot[presentation] coordinates {(6,19)}; % Symbole für Datentypen
      \addplot[interaction]  coordinates {(6,20)}; % Bedienreihenfolge von Funktionen
    \end{axis}
  \end{tikzpicture}
}

    \caption{Häufigkeit des Auftretens verschiedener Usability-Probleme.}
  \end{figure}

\end{frame}

\begin{frame}

  \begin{itemize}
    \item Verwirrungen bezüglich der Bedienreihenfolge der Funktionen.
  \end{itemize}

  \begin{figure}
    \begin{center}
      \includegraphics[width=0.95\textwidth]{assets/buffet-simple.png}
    \end{center}
    \caption{Bearbeitung einer Konvertierung im Block-Editor.}
  \end{figure}

\end{frame}

\begin{frame}
  % \frametitle{\currentsectionname{} (cont.)}

  \begin{figure}
    \centering
\begin{tikzpicture}
  \tikzset{icon/.style={outer xsep=4.5em, outer ysep=1em, minimum height=3em}}
  \tikzset{label/.style={font=\footnotesize}}
  \node [icon] (text)  at (0, 0)       {\includesvg[width=2em]{assets/bi-card-text.svg}};
  \node [icon] (int)   at (text.east)  {\includesvg[width=2em]{assets/bi-123.svg}};
  \node [icon] (float) at (int.east)   {\includesvg[width=2em]{assets/bi-123.svg}};
  \node [icon] (bool)  at (float.east) {\includesvg[width=2em]{assets/bi-toggles.svg}};
  \node [icon] (dt)    at (bool.east)  {\includesvg[width=2em]{assets/bi-calendar-week.svg}};
  \node [icon] (geom)  at (dt.east)    {\includesvg[width=2em]{assets/bi-pin-map.svg}};
  \node [label] at (text.south)  {Text};
  \node [label] at (int.south)   {Ganzzahl};
  \node [label] at (float.south) {Gleitkommazahl};
  \node [label] at (bool.south)  {Boolean};
  \node [label] at (dt.south)    {Datum/Zeit};
  \node [label] at (geom.south)  {Geometrie};
\end{tikzpicture}

    \caption{Die im Block-Editor verwendeten Symbole für Datentypen.}
  \end{figure}

  \begin{itemize}
    \item Verwechslung von Ganzzahl und Gleitkommazahl
    \item Symbol für Boolean wurde als Schaltflächen identifiziert
  \end{itemize}

  \Note{
    \item Verwechslung auch von Menschen mit Programmierkenntnissen
    \item Boolean-Problem Oft schon beim ersten Eindruck genannt
  }

\end{frame}

\begin{frame}
  % \frametitle{\currentsectionname{} (cont.)}

  \begin{itemize}
    \item Der Datentyp von Szenario-Feldern muss im Vorhinein bekannt sein und ausgewählt werden.
    \item Das Dropdown-Menü wurde nicht immer sofort gefunden.
  \end{itemize}

  \begin{figure}
    \begin{center}
      \includegraphics[width=0.95\textwidth]{assets/datatype-dropdown.png}
    \end{center}
    \caption{Dropdown-Menü zum Auswählen des Datentyps von Feldern.}
  \end{figure}

\end{frame}

\begin{frame}
  \frametitle{weitere aufgetretene Probleme}


  \begin{itemize}
    \item Interaktion
          \begin{itemize}
            \item Ersetzen von Einträgen
            \item Drag\&Drop
            \item Überladungen
          \end{itemize}
    \item Darstellung
          \begin{itemize}
            \item Konsistente Darstellung von Attributen
            \item Benennung von Speichern-Button und Auswahlmenü
          \end{itemize}
    \item Inhalt
          \begin{itemize}
            \item Details zu Parametern und Funktionen
            \item Zugriff auf Quelldaten
          \end{itemize}
  \end{itemize}

  \Note{
    \item Details: Längen und Breitengrad
  }

\end{frame}
