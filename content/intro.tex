\begin{frame}

  \begin{itemize}
    \item Große Komplexität im Bereich der Umweltdaten.\\($m \times n$ Transformationen)
    \item Starke Vereinfachung durch Sekundärsystem Simplex4TwIS.\\($m + n$ Transformationen)
  \end{itemize}

  \newtheorem{pro}{Problemstellung}
  \begin{pro}
    \begin{itemize}
      \item Weiterhin wird manuelle Eingabe von SQL benötigt.
      \item Kompliziert für Fachexpert:innen ohne technische Kenntnisse.
      \item Anfällig für Tippfehler.
      \item Struktur der Eingangsdaten muss bekannt sein.
    \end{itemize}
  \end{pro}

  \Note{
    \item Komplexität durch viele Primärsysteme, die Daten erfassen und heterogene Datensätze erzeugen
    \item viele Formate, sowohl in Eingangsdaten als auch Ausgabeformate
    \item Ausgabeformate müssen oft Daten aus verschiedenen Bereichen kombinieren, komplexe Datentransformationen ($m \times n$, jedes Eingangsformat könnte in jedem Ausgabeformat benötigt werden)
    \item Sekundärssystem Simplex4TwIS vereinfacht,
    \item harmonisierte Speicherung, ermöglicht wiederholbare Imports und Exports, $m$ Importe, $n$ Exporte \textrightarrow $m + n$
  }

\end{frame}

\begin{frame}

  \newtheorem{goal}{Zielsetzung}
  \begin{goal}
    \begin{itemize}
      \item Entwicklung eines Block-Editors, der die genannten Probleme angeht.
      \item Durchführung einer Usability-Studie, um die Effektivität der Herangehensweise zu überprüfen.
    \end{itemize}
  \end{goal}

\end{frame}
